\documentclass[12pt, letterpaper]{article}
\usepackage[utf8]{inputenc}

\usepackage{geometry}
\geometry{a4paper, total={6in,10in}}

\usepackage{palatino}
\fontfamily{ppl}\selectfont

\usepackage{csquotes}
\usepackage{amsmath}

\usepackage{graphicx}
\graphicspath{{images/}}

\title{\textbf{\Huge Relations and Functions \\ Part 1}}
\author{Aviral Janveja}
\date{2022}


\begin{document}

\maketitle

Much of the mathematics is about finding a pattern, a recognizable link between quantities that change. In this chapter, we will see how to link two sets and thus introduce \textbf{Relations} between them. We also learn about special relations which will qualify to be \textbf{Functions}. The concept of function is very important in mathematics since it captures the idea of a mathematically precise correlation between one quantity and another.


\section{Cartesian Product of Sets}
\begin{displayquote}
\textbf{Definition : Given two non-empty sets A and B. The cartesian product $A \times B$ is the set of all ordered pairs of elements from A and B. Denoted as $A \times B = \{ (a,b): a \in A \ \textnormal{and} \ b \in B \}$. The $ordered$ $pairs$ taken from any two sets $A$ and $B$ are pairs of elements written in brackets, grouped together in that particular order. That is, the set of all $(a,b)$ such that $a$ belongs to set $A$ and $b$ belongs to set $B$.}
\end{displayquote}
\textbf{For example},\\
Suppose $A$ is a set of colors and $B$ is a set of objects : 
\begin{displaymath}
A = \{red, blue\} \textnormal{ and } B = \{bag, coat\}
\end{displaymath}
Then the cartesian product $A \times B$ is obtained as follows : 
\begin{displaymath}
A \times B = \{(red,bag), (red, coat), (blue, bag), (blue, coat)\}
\end{displaymath}

\subsection{Remarks}
\begin{itemize}
    \item $(a,b)$ is not the same as $(b,a)$. Two ordered pairs are equal if and only if their corresponding first and second elements are equal. That is why it is called an \textbf{ordered} pair. Naturally, this implies that $A \times B \neq B \times A$.
    \item If there are $p$ elements in set $A$ and $q$ elements in set $B$, then there will be $p.q$ elements in the set $A \times B$. That is, $n(A \times B) = p.q$
    \item If either $A$ or $B$ is an empty set, then $A \times B$ will also be empty set.
    \item If $A$ and $B$ are non-empty sets and either $A$ or $B$ is an infinite set, then so is $A \times B$.
    \item $A \times A \times A = \{(a,b,c): a,b,c \in A\}$. Here $(a,b,c)$ is an ordered triplet.
\end{itemize}

\subsection{Example}
If $P = \{0,1\}$, form the set $P\times P \times P$. \\
$\Rightarrow$ Going step by first we can first calculate $P \times P$ \\
$\Rightarrow$ $P \times P = \{(0,0), (0,1), (1,0), (1,1)\}$ \\
$\Rightarrow$ And Therefore :
\begin{displaymath}
P \times P \times P = \{(0,0,0), (0,0,1), (0,1,0), (0,1,1), (1,0,0), (1,0,1), (1,1,0), (1,1,1)\}
\end{displaymath}

\subsection{Example}
$\mathbf{R}$ is the set of real numbers, what do the cartesian products $\mathbf{R} \times \mathbf{R}$ and $\mathbf{R} \times \mathbf{R} \times \mathbf{R}$ represent ?\\
$\Rightarrow$ The set $\mathbf{R}$ represents the points on a line.\\
$\Rightarrow$ The set $\mathbf{R} \times \mathbf{R} = \{(x,y): x,y \in \mathbf{R}\}$ represents the coordinates of all the points in two-dimensional space.\\
$\Rightarrow$ The set $\mathbf{R} \times \mathbf{R} \times \mathbf{R} = \{(x,y,z): x,y,z \in \mathbf{R}\}$ represents the coordinates of all the points in three-dimensional space.

\subsection{Example}
If $A \times B = \{(p,q),(p,r),(m,q),(m,r)\}$, find $A$ and $B$. \\
$A$ = set of first elements = $\{p,m\}$\\
$B$ = set of second elements = $\{q,r\}$

\section{Relations}
\begin{displayquote}
\textbf{Definition : A relation $R$ from a non-empty set $A$ to a non-empty set $B$ is a subset of the cartesian product $A \times B$. This subset is obtained by describing a relationship between the first element and the second element of the ordered pairs in $A \times B$. The second element is called the  $image$ of of the first element.\\
The set of all first elements of the ordered pairs in relation $R$ is called the $domain$ of the relation and the set of all second elements is called the $range$ of the relation $R$. The whole set $B$ is called the $codomain$ of the relation $R$.}
\end{displayquote}
\textbf{For example},\\
 Consider a set $P = \{9,4,25\}$ and $Q = \{5,3,2,1,-2,-3,-5\}$\\
 and the following relation $R = \{(x,y) : x = y^2, x \in P, y \in Q\}$\\
The above relation $R$ which is a subset of $P \times Q$ can also be written as :\\
$R = \{(9,3), (9,-3), (4,2), (4,-2), (25,5), (25,-5)\}$\\
As per the definition, \\
The domain of the above relation is $\{4,9,25\}$\\
Whereas, the range of this relation is $\{-2,2,-3,3,-5,5\}$\\
And of course, the set $Q$ is the co-domain of this relation.\\
\textbf{Remark} : 
\begin{enumerate}
    \item A relation $R$ from set $A$ to $A$ is also stated in short as a ``relation on set $A$" or a ``relation in set $A$".
    \item A relation is essentially a certain set. Therefore just like any set, it may be represented by the roster or the set builder method.
    \item Visually, an arrow diagram can be used to represent a relation. As shown in the reference text 1.
\end{enumerate}

\subsection{Note}
\begin{displayquote}
\textbf{The total number of relations that can be defined from a set $A$ to a set $B$ is obviously the number of possible subsets of $A \times B$.\\
So, if $n(A) = p$ and $n(B) = q$, then $n(A \times B) = pq$. Then the total number of relations is $2^{pq}$.}
\end{displayquote}
\textbf{How come the number of relations or subsets is equal to $2^{pq}$ and how to write down these subsets ?}\\
Think of it in terms of binary digits. With one bit, we have two possible states - 0 \& 1. With two bits, we have 4 possible states - $(00,01,10,11)$ and similarly 8 states with 3 bits, 16 states with 4 bits and so on.\\
The number of possible subsets for a set can be calculated in a similar fashion. For example, taking a set $A = \{2,3\}$ with two elements, the subsets can be computed as follows : \\
$\Rightarrow$ $00$ = A subset with neither first nor second element = $\{\emptyset\}$\\
$\Rightarrow$ $01$ = A subset with only the second element = $\{3\}$\\
$\Rightarrow$ $10$ = A subset with only the first element = $\{2\}$\\
$\Rightarrow$ $11$ = A subset with both the elements = $\{2,3\}$\\
In general for any set $A$, with $n(A) = m$, we will have in total $2^m$ subsets or relations. The set containing all the $2^m$ subsets of $A$ is obviously called the \textbf{power set} of $A$ as seen in the chapter on sets. Finally, these subsets can be written down using bit analogy way shown above.

\section{Functions}
\begin{displayquote}
\textbf{A relation from a set $A$ to a set $B$ is said to be a function if every element of set $A$ has one and only one image in set $B$. The function $f$ from set $A$ to set $B$ is denoted by $f : A \rightarrow B$ where $f(a) = b$ such that $a \in A$ and $b \in B$. Here, $b$ is called the image of $a$ under $f$.}
\end{displayquote}
In other words, a function $f$ is a special type of relation for which, the domain is set $A$ (every element in set $A$) and no two distinct ordered pairs $(a,b) \in f$ have the same first element (one and only one image in set $B$). The term \textbf{map} or \textbf{mapping} is sometimes used to denote a function.


\subsection{Example}
$\mathbf{N}$ is the set of natural numbers. A relation $R$ is defined on $\mathbf{N}$ such that $R = \{(x,y) : y=2x \textnormal{ where } x,y \in \mathbf{N}\}$\\
\textbf{Question} : What is the domain, codomain and range of $R$ ? Is this relation a function ?\\
\textbf{Solution} : The Domain of $R$ is the set of natural numbers. The codomain is also $\mathbf{N}$. The range is the set of even natural numbers. Since every natural number has one and only one image as per the defined relation, therefore this relation is a function.\\
\textbf{Note} : A $function$ whose $range$ is real valued is called a $real$ $valued$ $function$. Further, if its $domain$ is also real valued, it is called a $real$ $function$.

\section{Some Common Functions}

\subsection{Identity Function}
Let $\mathbf{R}$ be the set of real numbers. The identity function is defined as $f : \mathbf{R} \rightarrow \mathbf{R}$, where $y = f(x) = x$ for each $x \in \mathbf{R}$. The $domain$ and $range$ of $f$ are $\mathbf{R}$.

\subsection{Constant Function}
The constant function is defined as $f : \mathbf{R} \rightarrow \mathbf{R}$, where $y = f(x) = c$ and $c$ is a constant. Here the $domain$ of $f$ is $\mathbf{R}$ and its $range$ is $\{c\}$.

\subsection{Polynomial Function}
A function $f : \mathbf{R} \rightarrow \mathbf{R}$ is said to be a polynomial function defined as $y = f(x) = a_0 + a_1x + a_2x^2 + ... + a_nx^n$, where $n$ is a non-negative integer and $a_0, a_1, a_2...a_n \in \mathbf{R}$.                

\subsection{Rational Function}
These are functions of the type $\frac{f(x)}{g(x)}$, where $f(x)$ and $g(x)$ are polynomial functions of $x$ defined in a domain, where $g(x) \neq 0$.\\
\textbf{For example}, consider the real valued function $f : \mathbf{R} - \{0\} \rightarrow \mathbf{R}$ defined by : 
\begin{displaymath}
f(x) = \frac{1}{x}
\end{displaymath}

\subsection{Modulus Function}
The function $f : \mathbf{R} \rightarrow \mathbf{R}$ denoted by $f(x) = |x|$ is called the modulus function. It is defined as : 
\begin{displaymath}
|x| = \begin{cases}
x & \text{if } x \geq 0 \\
-x & \text{if } x < 0
\end{cases}
\end{displaymath}
The domain of the modulus function is the set of real numbers $\mathbf{R}$, whereas the range is the set of all non-negative real numbers.

\section{Algebra of Real Functions}
\begin{enumerate}
    \item \textbf{Addition} : Let $f$ and $g$ be two functions, then we define $(f+g) (x) = f(x) + g(x)$
    \item \textbf{Subtraction} : Let $f$ and $g$ be two functions, then we define $(f-g) (x) = f(x) - g(x)$
    \item \textbf{Multiplication by Scalar} : Let $f$ be a function and $\alpha$ be a real number. Then we define $(\alpha f)(x) = \alpha f(x)$
    \item \textbf{Multiplication} : Let $f$ and $g$ be two functions, then we define $(f.g) (x) = f(x) . g(x)$
    \item \textbf{Quotient} : Let $f$ and $g$ be two functions, then quotient $f$ by $g$ is defined by $\frac{f}{g}(x) = \frac{f(x)}{g(x)}$, provided $g(x) \neq 0$.
\end{enumerate}

\section{Solved Exercises}
\subsection{}
Let $f(x) = x^2$ and $g(x) = 2x+1$ \\
Find : $(f+g)(x)$, $(f-g)(x)$, $(f.g)(x)$ and $\frac{f}{g}(x)$\\
\textbf{Solution} :\\
$(f+g) (x) = f(x) + g(x) = x^2 + 2x + 1$ \\
$(f-g) (x) = f(x) - g(x) = x^2 - 2x -1$ \\
$(f.g) (x) = f(x) . g(x) = 2x^3 + x^2$ \\
$\frac{f}{g}(x) = \frac{f(x)}{g(x)} = \frac{x^2}{2x+1}$, $x \neq -1/2$

\subsection{}
Let $R$ be a relation on the set of rational numbers $\mathbf{Q}$ defined by : 
\begin{displaymath}
R = \{(a,b) : a-b \in \mathbf{Z} \textnormal{ where }a,b \in \mathbf{Q}\}
\end{displaymath}
Show that : 
\begin{enumerate}
    \item $(a,a) \in R$ for all $a \in \mathbf{Q}$
    \item $(a,b) \in R$ implies that $(b,a) \in R$
    \item $(a,b) \in R$ and $(b,c) \in R$ implies that $(a,c) \in R$
\end{enumerate}
\textbf{Solution} : 
\begin{enumerate}
    \item For any rational number $a$, $a-a = 0$, which is obviously an integer. Therefore, it follows that $(a,a) \in R$ for all $a \in \mathbf{Q}$
    \item $(a,b) \in R$ implies that $a-b \in \mathbf{Z}$. So, If $a-b$ is an integer then it follows naturally that $b-a$ which is simply $-(a-b)$ is an integer as well. Therefore $(b,a) \in R$
    \item $(a,b) \in R$ and $(b,c) \in R$ implies that $a-b \in \mathbf{Z}$ and $b-c \in \mathbf{Z}$. The sum of two integers will be an integer as well. Adding the two $(a-b) + (b-c) = a-c \in \mathbf{Z}$. Therefore, $(a,c) \in R$  
\end{enumerate}

\subsection{}
Let $f = \{(1,1), (2,3), (0,-1), (-1,-3)\}$ be a linear function on the set of integers $\mathbf{Z}$, then find $f(x)$.\\
\textbf{Solution} : Since $f$ is a \textbf{linear function} which by definition are of the form : 
\begin{displaymath}
f(x) = mx + c
\end{displaymath}
Where $m$ and $c$ are constants. Also since $(1,1) \in f$ and $(0,-1) \in f$ , substituting these values $(x,y)$ in $y = f(x)$ we have 
\begin{displaymath}
1 = f(1) = m(1) + c
\end{displaymath}
\begin{displaymath}
-1 = f(0) = m(0) + c
\end{displaymath}
Solving the two equations $(1 = m + c)$ and $(-1 = c)$ from above, we get $m = 2$, $c = -1$ and thus $f(x)$ is found : 
\begin{displaymath}
f(x) = 2x-1
\end{displaymath}  

\section{References}
\begin{enumerate}
    \item Class 11 - Chapter 2 : Relations and Functions.\\ 
    NCERT Mathematics Textbook, Version 2020-21.\\
    As per Indian National Curriculum Framework 2005.
\end{enumerate}

\end{document}
