\documentclass[12pt, letterpaper]{article}
\usepackage[utf8]{inputenc}

\usepackage{geometry}
\geometry{a4paper, total={6in,10in}}

\usepackage{palatino}
\fontfamily{ppl}\selectfont

\usepackage{csquotes}
\usepackage{amsmath}

\usepackage{graphicx}
\graphicspath{{images/}}

\title{\textbf{\Huge Probability \\ Part 1}}
\author{Aviral Janveja}
\date{2022}


\begin{document}

\maketitle

The science of probability began as a \textbf{study of uncertainty} in games of chance. The earliest known records on probability are found in texts like the Rig Veda and later quite popularly in Mahabharata. Today, it has been elevated to the rank of one of the most important subjects of human knowledge.

\section{Probability - An Experimental Approach}

We get glimpses of \textbf{uncertainty} in our everyday lives. Come to think of it, any game has an element of uncertainty in it, whether Tennis where you cannot be sure of the other player's moves, a team sport like Football with multiple variables or Martial Arts.\\
For instance, we are never really sure of the outcome when tossing a coin or throwing a dice. There is always that element of uncertainty, an element of chance so to speak.\\
We can treat these games as experiments and in turn observe their outcomes. Here, we will try to measure the chance of occurrence of a particular outcome in an experiment numerically.

\subsection{Coin Toss Experiment}
Our \textbf{Experiment} in this case is tossing of a coin, where we toss a coin and observe the \textbf{outcome}.\\
Each toss of a coin in this case is called a \textbf{trial}. A trial is an action which results in one or several outcomes.\\
Further, an \textbf{event} for an experiment is the collection of one or more outcomes of the experiment. For instance, in our coin toss experiment, ``getting a tail" is an event with outcome ``tail" or while throwing a dice, the event ``getting an even number" is an event consisting outcomes 2, 4 and 6.

\subsection{Definition}
Based on what we observe as the outcomes of our trials, we find the \textbf{experimental} or \textbf{empirical probability}. The empirical probability of an event $E$ happening is given by : 
\begin{displaymath}
    \text{P(E)} = \frac{\text{Number of trials in which E happened}}{\text{Total number of trials}}
\end{displaymath}
So, the empirical probability is based on the number of trials undertaken and the number of times the event you are looking for comes up.

\section{Examples}

\subsection{}
A coin is tossed 1000 times, and the following outcomes are observed :\\
Head : 455, Tail: 545.\\
Compute the probability for both events ``getting a head" and ``getting a tail".\\
\textbf{Solution :}\\
Let us call the event of getting a head as H and that of getting a tail as T. Then as per the definition of experimental probability : 
\begin{displaymath}
    P(H) = \frac{455}{1000} = 0.455
\end{displaymath}
Similarly, the probability of getting a tail will be : 
\begin{displaymath}
    P(T) = \frac{545}{1000} = 0.545
\end{displaymath}

\subsection{}
Two coins were simultaneously tossed 500 times and the following events were recorded based on the outcomes observed : \\
Two Heads (Head Head) = 105\\
One Head (Head Tail \& Tail Head) = 275\\
No Head (Tail Tail) = 120\\
Find the probability of occurrence of each of these events.\\
\textbf{Solution :}\\
Denoting the events of getting two heads, one head and no head as HH, H and T respectively. We can then calculate the empirical probability as follows : 
\begin{align*}
    P(HH) = \frac{105}{500} = 0.21 \\
    \\
    P(H) = \frac{275}{500} = 0.55 \\
    \\
    P(T) = \frac{120}{500} = 0.24
\end{align*}

\textbf{Note :} The probability of an event always lies in-between 0 and 1, that is : 
$$\mathbf{0 \leq P(E) \leq 1}$$


\section{References}
\begin{enumerate}
    \item Class 9 - Chapter 15 : Probability.\\ 
    NCERT Mathematics Textbook, Version 2020-21.\\
    As per Indian National Curriculum Framework 2005.
\end{enumerate}

\end{document}