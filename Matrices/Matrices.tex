\documentclass[12pt, letterpaper]{article}
\usepackage[utf8]{inputenc}

\usepackage{geometry}
\geometry{a4paper, total={6in,10in}}

\usepackage{palatino}
\fontfamily{ppl}\selectfont

\usepackage{csquotes}
\usepackage{amsmath}

\usepackage{graphicx}
\graphicspath{{images/}}

\title{\textbf{\Huge Matrices}}
\author{Aviral Janveja}
\date{2022}


\begin{document}

\maketitle

The concept of \textbf{Matrix} evolved through an attempt to obtain simpler and more compact methods of solving systems of linear equations.\\
Matrices are used as a representation for the coefficients in systems of linear equations. They simplify our work to a great extent and are therefore useful in various branches of science and mathematics from physics, cryptography, genetics, economics and so on.


\section{What is a Matrix ?}
\begin{displayquote}
\textbf{ Definition : A matrix is an ordered rectangular array of numbers or functions. The numbers or functions are called the elements or the entries of the matrix.}
\end{displayquote}
\textbf{For example},\\
The situation that ``Aviral has 1 pen and 2 note books, Siddharth has 2 pens and 1 notebook and Shreyas has 2 pens and 2 notebooks" can be represented in matrix form as follows : 
\begin{displaymath}
\begin{split}
A = \begin{bmatrix}
1 & 2 & 2\\
2 & 1 & 2
\end{bmatrix}
\end{split}
\quad \textnormal{or} \quad
\begin{split}
    A = \begin{bmatrix}
1 & 2 \\
2 & 1 \\
2 & 2
\end{bmatrix}
\end{split}
\end{displaymath}
We denote matrices with capital letters. In the above matrix $A$, the horizontal lines of elements are called \textbf{rows} and the vertical lines of elements are called \textbf{columns}.

\subsection{Order of a Matrix}
\begin{displayquote}
\textbf{ Definition : A matrix having $m$ rows and $n$ columns is said to have order $m \times n$, read as ``m by n" matrix.}
\end{displayquote}
In the general a $m \times n$ matrix has the following representation :
\begin{displaymath}
X = \begin{bmatrix}
a_{11} & a_{12} & a_{13} & ... & a_{1n}\\
a_{21} & a_{22} & a_{23} & ... & a_{2n}\\
a_{31} & a_{32} & a_{33} & ... & a_{3n}\\
a_{41} & a_{42} & a_{43} & ... & a_{4n}\\
.\\
.\\
.\\
a_{m1} & a_{m2} & a_{m3} & ... & a_{mn}
\end{bmatrix}_{m \times n}
\end{displaymath}
So, here we see that $X$ is a matrix of order $m \times n$. The number of elements in an $m \times n$ matrix will obviously be equal to $m$ multiplied by $n$.\\
The individual elements of the matrix are represented by $a$, where $a_{ij}$ refers to the element lying in the $i^{th}$ row and $j^{th}$ column.\\
\textbf{For example}, \\
$a_{43}$ is an element of $X$ lying in the $4^{th}$ row and $3^{rd}$ column.


\section{Types of Matrices}

\subsection{Column Matrix}
\begin{displayquote}
\textbf{Definition : A matrix is said to be a column matrix if it has only one column.}
\end{displayquote}
\textbf{For example},\\
The following is a column matrix of order $3 \times 1$ : 
\begin{displaymath}
A = \begin{bmatrix}
1 \\
2 \\
3
\end{bmatrix}
\end{displaymath}

\subsection{Row Matrix}
\begin{displayquote}
\textbf{Definition : A matrix is said to be a row matrix if it has only one row.}
\end{displayquote}
\textbf{For example},\\
The following is said to be a row matrix of order $1 \times 3$ : 
\begin{displaymath}
A = \begin{bmatrix}
1 & 2 & 3\\
\end{bmatrix}
\end{displaymath}

\subsection{Square Matrix}
 \begin{displayquote}
\textbf{Definition : A matrix in which the number of rows and columns are equal is called a square matrix.}
\end{displayquote}
Thus, in case of a square matrix m = n. Hence, it will be simply called a square matrix of order n.\\
\textbf{For example},\\
The following is a square matrix of order 3 : 
 \begin{displaymath}
 A = \begin{bmatrix}
1 & 2 & 3 \\
4 & 5 & 6 \\
7 & 8 & 9
\end{bmatrix}
 \end{displaymath}

\subsection{Diagonal Matrix}
\begin{displayquote}
\textbf{Definition : A square matrix is called a diagonal matrix if all its non diagonal elements are zero. meaning $a_{ij} = 0$ when $i \neq j$.}
\end{displayquote}
\textbf{For example},\\
The following is a diagonal matrix of order 3 : 
\begin{displaymath}
A = \begin{bmatrix}
1 & 0 & 0 \\
0 & 5 & 0 \\
0 & 0 & 9
\end{bmatrix}
\end{displaymath}

\subsection{Scalar Matrix}
\begin{displayquote}
\textbf{Definition : A diagonal matrix is called a scalar matrix if all its diagonal elements are equal.}
\end{displayquote}
\textbf{For example},\\
The following is a scalar matrix of order 3 : 
\begin{displaymath}
A = \begin{bmatrix}
5 & 0 & 0 \\
0 & 5 & 0 \\
0 & 0 & 5
\end{bmatrix}
\end{displaymath}

\subsection{Identity Matrix}
\begin{displayquote}
\textbf{Definition : A scalar matrix is called an Identity matrix if all its diagonal elements are equal to 1. We denote the identity matrix by $I$, where the order of matrix is clear from the context :}
\end{displayquote}
\begin{displaymath}
I = \begin{bmatrix}
1 & 0 & 0 \\
0 & 1 & 0 \\
0 & 0 & 1
\end{bmatrix}
\end{displaymath}

\subsection{Zero Matrix}
\begin{displayquote}
\textbf{Definition : A matrix is called a zero matrix or null matrix if all its elements are zero. We denote the zero matrix by $O$, where the order of matrix is clear from the context :}
\end{displayquote}
\begin{displaymath}
O = \begin{bmatrix}
0 & 0 & 0 \\
0 & 0 & 0 \\
0 & 0 & 0
\end{bmatrix}
\end{displaymath}

\section{Equality of Matrices}
\begin{displayquote}
\textbf{Definition : Two matrices $A$ and $B$ are said to be equal if they are of the same order and each element of $A$ is equal to the corresponding element of $B$. That is $a_{ij} = b_{ij}$ for all $i$ and $j$.} 
\end{displayquote}
\textbf{For example},\\
The following are an example of equal matrices : 
$\begin{bmatrix}
  2 & 3\\ 
  0 & 1
\end{bmatrix} = \begin{bmatrix}
  2 & 3\\ 
  0 & 1
\end{bmatrix}$.


\section{Operations on Matrices}

\subsection{Addition of Matrices}
\begin{displayquote}
\textbf{Definition : The sum of two matrices $A$ and $B$ is a matrix $C$, which is obtained by adding corresponding elements of $A$ and $B$. That is, $c_{ij} = a_{ij} + b_{ij}$ for all $i$ and $j$. Furthermore, the two matrices have to be of the same order.} 
\end{displayquote}

\textbf{For example},\\
Given, 
\begin{displaymath}
A = \begin{bmatrix}
\sqrt{3} & 1 & -1\\
2 & 3 & 0
\end{bmatrix}
\end{displaymath} 
And 
\begin{displaymath}
B = \begin{bmatrix}
2 & \sqrt{5} & 1\\
-2 & 3 & 1/2
\end{bmatrix}
\end{displaymath}
Then,
\begin{displaymath}
C = A + B = \begin{bmatrix}
\sqrt{3}+2 & 1+\sqrt{5} & 0\\
0 & 6 & 1/2
\end{bmatrix}
\end{displaymath}
It is important to note that, if $A$ and $B$ are not of the same order, then $A+B$ is not defined.\\
The addition of matrices satisfies the following \textbf{properties} :
\begin{enumerate}
    \item \textbf{Commutative} : $A+B = B+A$
    \item \textbf{Associative} : $(A+B)+C = A+(B+C)$
    \item \textbf{Additive Identity} : $A+O = O+A = A$. In other words, Null Matrix $O$ is the additive identity of matrix addition. 
    \item \textbf{Additive Inverse} : $A + (-A) = (-A) + A = O$. So, $-A$ is the additive inverse of A.
\end{enumerate}
\subsection{Scalar Multiplication of Matrices}
\begin{displayquote}
\textbf{Definition : Given a matrix $A$ and a scalar $k$ then $kA$ is another matrix which is obtained by multiplying each element of $A$ by the scalar $k$. That is $a_{ij}$ becomes $ka_{ij}$ for all $i$ and $j$.} 
\end{displayquote}
\textbf{For example},\\
If,
\begin{displaymath}
A = \begin{bmatrix}
3 & 1 & 1.5\\
5 & 7 & -3\\
2 & 0 & 5
\end{bmatrix}
\end{displaymath}
Then, 
\begin{displaymath}
2A = \begin{bmatrix}
6 & 2 & 3\\
10 & 14 & -6\\
4 & 0 & 10
\end{bmatrix}
\end{displaymath}
\textbf{Negative} of a matrix is denoted by $-A$. We define 
\begin{displaymath}
-A = (-1).A
\end{displaymath}
The scalar multiplication of matrix, satisfies the following \textbf{properties} :
\begin{enumerate}
    \item $k(A+B) = kA + kB$
    \item $(k+l)A = kA +lA$
\end{enumerate}
Where, $A$ and $B$ are matrices of the same order whereas $k$ and $l$ are scalars.

\subsection{Difference of Matrices}
\begin{displayquote}
\textbf{Definition : The difference of two matrices $A$ and $B$ is a matrix $C$, which is obtained by subtracting corresponding elements of $A$ and $B$. That is, $c_{ij} = a_{ij} - b_{ij}$ for all $i$ and $j$. Furthermore, the two matrices have to be of the same order.} 
\end{displayquote}
\textbf{For example},\\
If, 
\begin{displaymath}
A = \begin{bmatrix}
1 & 2 & 3\\
2 & 3 & 1
\end{bmatrix}
\end{displaymath} 
And 
\begin{displaymath}
B = \begin{bmatrix}
2 & 4 & 6\\
4 & 6 & 2
\end{bmatrix}
\end{displaymath}
Then,
\begin{displaymath}
 2A - B = \begin{bmatrix}
0 & 0 & 0\\
0 & 0 & 0
\end{bmatrix}
\end{displaymath}

\subsection{Multiplication of Matrices}
\begin{displayquote}
\textbf{Definition : Firstly, The product of two matrices $A$ and $B$ is defined if the number of columns of $A$ is equal to the number of rows of $B$.\\ 
Then, the product of the matrices $A$ and $B$ is the matrix $C$, whose order is given by the number of rows of $A$ and number of columns of $B$.\\
Finally, to obtain the elements $c_{ij}$ of $C$, we take the $i^{th}$ row of $A$ and $j^{th}$ column of $B$, multiply them element-wise and take the sum these products.}  
\end{displayquote}
\textbf{For example},\\
If
\begin{displaymath}
 A = \begin{bmatrix}
1 & -1 & 2\\
0 & 3 & 4
\end{bmatrix}_{2 \times 3}
\end{displaymath} 
And 
\begin{displaymath}
B = \begin{bmatrix}
2 & 7\\
-1 & 1\\
5 & -4
\end{bmatrix}_{3 \times 2}
\end{displaymath}
$\Rightarrow$ Then, firstly product $AB$ is defined as number of columns of $A$ is same as the number of rows of $B$.\\
$\Rightarrow$ Secondly, the resultant matrix $C$ has order $2 \times 2$, given by the number of rows of $A$ and number of columns of $B$.\\
$\Rightarrow$ Finally, the resultant matrix can be written as : 
\begin{displaymath}
AB = C = \begin{bmatrix}
c_{11} & c_{12}\\
c_{21} & c_{22}
\end{bmatrix}_{2 \times 2}
\end{displaymath}
Where the elements $c_{ij}$ of $C$ can be computed as specified in the definition above.\\
$\Rightarrow$ For instance, to calculate $c_{21}$, we take the second row of $A$ and the first column of $B$, multiply their corresponding elements and take the sum. The elements of $C$ are computed accordingly and shown below for illustration :  
$$c_{11} = (1)(2) + (-1)(-1) + (2)(5) = 13$$
$$c_{12} = (1)(7) + (-1)(1) + (2)(-4) = -2$$
$$c_{21} = (0)(2) + (3)(-1) + (4)(5) = 17$$ 
$$c_{22} = (0)(7) + (3)(1) + (4)(-4) = -13$$
Therefore, 
\begin{displaymath}
AB = C = \begin{bmatrix}
13 & -2\\
17 & -13
\end{bmatrix}_{2 \times 2}
\end{displaymath}
The multiplication of matrices satisfies the following \textbf{properties} : 
\begin{enumerate}
    \item \textbf{Non-commutative} : Even if products $AB$ and $BA$ are both defined, It is not necessary that $AB$ equals $BA$ that is $AB \neq BA$.\\
    \textbf{Remark} : Multiplication of diagonal matrices of the same order will be commutative.
    \item \textbf{Associative} : $(AB)C = A(BC)$, whenever both sides of the equality are defined.
    \item \textbf{Distributive} : $A(B+C) = AB + AC$ and $(A+B)C = AC + BC$, whenever both sides of the equality are defined.
    \item \textbf{Multiplicative Identity} : For every square matrix $A$, there exists an Identity matrix of the same order such that $IA = AI = A$.
\end{enumerate}

\section{Transpose of a Matrix}
\begin{displayquote}
\textbf{Definition : Let $A$ be a $m \times n$ matrix, then the matrix obtained by interchanging the rows and columns of $A$ is called the transpose of $A$. That is $a_{ij} \rightarrow a_{ji}$ for all $i$ and $j$. This new $n \times m$ matrix is denoted by $A'$ or $A^T$}.
\end{displayquote}
\textbf{For example}, \\
If 
\begin{displaymath}
A = \begin{bmatrix}
3 & 5\\
\sqrt{3} & 1\\
0 & -1
\end{bmatrix}_{3 \times 2}
\end{displaymath}
Then
\begin{displaymath}
A' = \begin{bmatrix}
3 & \sqrt{3} & 0\\
5 & 1 & -1
\end{bmatrix}_{2 \times 3}
\end{displaymath}
The transpose of matrices, satisfies the following \textbf{properties} :
\begin{enumerate}
    \item $(A')' = A$
    \item $(kA)' = kA'$ (where $k$ is any constant)
    \item $(A + B)' = A' + B'$
    \item $(AB)' = B'A'$
\end{enumerate}

\section{Symmetric and Skew Symmetric Matrices}
\begin{displayquote}
\textbf{Definition : A square matrix $A$ is said to be a symmetric if $A' = A$, that is $a_{ij} = a_{ji}$ for all $i,j$. Further, a square matrix $A$ is said to be skew-symmetric if $A' = -A$, that is $a_{ij} = -a_{ji}$ for all $i,j$.}
\end{displayquote}
\textbf{For example}, the following is a symmetric matrix : 
\begin{displaymath}
A = \begin{bmatrix}
3 & 2 & 3\\
2 & -3 & -1\\
3 & -1 & 1
\end{bmatrix}
\end{displaymath}
Whereas, the following is a skew-symmetric matrix : 
\begin{displaymath}
B = \begin{bmatrix}
0 & e & f\\
-e & 0 & g\\
-f & -g & 0
\end{bmatrix}
\end{displaymath}
Notice, as per the definition of a skew-symmetric matrix, $a_{ij} = -a_{ji}$ for all $i,j$. Therefore, when $i=j$ : \\
$\Rightarrow$ We have $a_{ii} = -a_{ii}$\\
$\Rightarrow$ $2a_{ii} = 0$\\
$\Rightarrow$ $a_{ii} = 0$ for all $i$\\
This means that the diagonal elements of a skew-symmetric matrix are always zero.

\subsection{Theorem 1}
\begin{displayquote}
\textbf{For any square matrix $A$ with real number entries, $A+A'$ is a symmetric matrix and $A-A'$ is a skew symmetric matrix.}
\end{displayquote}
\textbf{Proof :}\\
The first part of the above theorem is proven if we show $(A + A')' = A + A'$\\
Taking $(A + A')'$\\
$ = A' + (A')'$\\
$= A' + A$\\
$= A + A'$
\smallskip

For the second part, we need to show that $(A - A')' = -(A - A')$\\
Let us take $(A - A')'$\\
$= A' - (A')'$\\
$ = A' - A$\\
$= -(A - A')$\\
Hence Proved.


\subsection{Theorem 2}
\begin{displayquote}
\textbf{Any square matrix can be expressed as the sum of a symmetric and a skew-symmetric matrix.}
\end{displayquote}
\textbf{Proof :}\\
Let $A$ be a square matrix, we can thus write -\\
$2A = A + A$\\
$2A = A + A + O$\\
$2A = A + A + A' - A'$\\
$2A = (A+A') + (A-A')$\\
$2A$ has been thus expressed as the sum of a symmetric and a skew-symmetric matrix and thus $A$ as well, can be simply written as - \\
$ A = 1/2(A+A') + 1/2(A-A')$\\
Hence Proved.

\section{Elementary Transformations of a Matrix}
There are three main elementary transformations of a matrix : 
\begin{enumerate}
    \item The interchange of any two rows or two columns. Denoted as $R_i \leftrightarrow R_j$ or $C_i \leftrightarrow C_j$
    \item The multiplication of elements of any row or column by a non-zero number. Denoted by $R_i \rightarrow kR_i$ or $C_i \rightarrow kC_i$ where $k \neq 0$.
    \item The addition to the elements of any row or column, the corresponding elements of any other row or column multiplied by any non-zero number. Denoted by $R_i \rightarrow R_i + kR_j$ or $C_i \rightarrow C_i + kC_j$ where $k \neq 0$.
\end{enumerate}

\section{Invertible Matrices}
\begin{displayquote}
\textbf{Definition : If $A$ is a square matrix and if there exists another square matrix $B$ of the same order, such that $AB = BA = I$, then $B$ is called the inverse of $A$ and is denoted by $A^{-1}$. In that case $A$ is said to be invertible.}
\end{displayquote}
\textbf{For example},\\
Let 
\begin{displaymath}
A = \begin{bmatrix}
2 & 3\\
1 & 2
\end{bmatrix} \textnormal{and} \ B = \begin{bmatrix}
2 & -3\\
-1 & 2
\end{bmatrix}
\end{displaymath}
We see that $AB = BA = I$. Thus $B$ is the inverse of $A$ and $A$ is the inverse of $B$, that is $A = B^{-1}$ and $B = A^{-1}$.\\
The following points are to be noted regarding inverse of a matrix : 
\begin{itemize}
    \item A rectangular matrix does not possess an inverse. 
    \item Inverse of a square matrix, if it exists, is unique.
    \item If $A$ and $B$ are invertible matrices of the same order, then $(AB)^{-1} = B^{-1}A^{-1}$.
\end{itemize}

\section{References}
\begin{itemize}
    \item Mathematics - Class 12, Chapter 3 - Matrices. NCERT Textbook India, version 2020-21. As per National Curriculum Framework 2005.
\end{itemize}
\end{document}
